\documentclass[12pt]{article}

% Generelles
\usepackage[utf8]{inputenc}
\usepackage[ngerman]{babel}
\usepackage[T1]{fontenc}
\usepackage{lmodern}
% Mathematik
\usepackage{amsmath}
\usepackage{amssymb}
% Zitate
\usepackage{epigraph}

\title{Freiheit}
\author{Luca Leon Happel}
\date{Fr 20 Dez 2019 18:19:19 CET}

\begin{document}
	\maketitle
	\tableofcontents
	\newpage

	\section{Einleitung}
		\epigraph{
			Wollen befreit: das ist die wahre Lehre von
			Wille und Freiheit - so lehrt sie euch Zarathustra.
		}{
			\textit{Friedrich Nietzsche\\- Also sprach Zarathustra}
		}
		``Die Freiheit'', ein Begriff, welcher recht einfach zu sein
		scheint, doch hinter welchem viele Fragen stecken, welche
		zu teils tief in unser Verständnis der Realität greifen.
		\begin{itemize}
			\item Was ist Freiheit
			\item Wodurch entsteht Freiheit
			\item Existiert Freiheit
			\item Was macht Freiheit
			\item Sind Menschen frei
		\end{itemize}
		Dies sind einige der Fragen, welche ich in diesem Artikel
		versuchen werde zu beantworten. Die Methodik dahinter ist, dass
		ich ein Modell für die Freiheit aufbauen werde, welches mein
		Verständnis von Freiheit darstellt. Wenn dieses Modell
		konstruiert ist, kann ich mit dessen Hilfe Antworten auf die
		zuvor gestellten Fragen herleiten und somit mehr über unsere
		Realität und Existenz herausfinden.
		Die Grundidee dabei ist, dass ich eine mathematische
		Herangehensweise wähle um dieses Modell zu konstruieren.

	\section{Wahlen und Einschränkungen}
		\subsection{Freiheit und Belieben}
			Sei $x\in X$ frei wählbar $\Leftrightarrow$
			Sei $x\in X$ beliebig\\
			$\Rightarrow$ frei wählbar $=$ beliebig
		\subsection{Gleichheit und Belieben}
			$\forall x \in X : f(x)$\\
			Suche ein $x$ mit $f(x)$.\\
			Da dies für alle $x$ gegeben ist, ist $x$ beliebig\\
			$\Rightarrow x$ ist frei unter der Auswahl $\{x\in X | f(x)\}$
		\subsection{Entscheidungsraum}
			Sei $X$ eine beliebige Menge\\
			$X$ ist ein Entscheidungsraum
		\subsection{Definition: Einschränkung}
			Eine \textbf{Einschränkung} $E:X\Rightarrow \{wahr, falsch\}$ eines
			Entscheidungsraumes $X$ ist eine
			Relation der Elemente von $X$ auf Werte wahr oder falsch
			sein können.\\
			\subsubsection{Kurznotation}
				\begin{itemize}
					\item $E(x) \Leftrightarrow E(x)=0 \Leftrightarrow E(x)$ ist wahr
					\item $\neg E(x) \Leftrightarrow E(x)\neq 0 \Leftrightarrow E(x)$ ist falsch
				\end{itemize}
			\subsubsection{freie/leere Einschränkung}
				Sei $E$ eine Einschränkung\\
				$E$ ist freie Einschränkung $\Leftrightarrow$ $E(x)\> \forall x\in X$
			\subsubsection{totale Einschränkung}
				Sei $E$ eine Einschränkung\\
				$E$ ist totale Einschränkung $\Leftrightarrow$ $\neg E(x)\> \forall x\in X$
		\subsection{Definition: Möglichkeitsraum}
			Sei $X$ eine Entscheidungsraum.\\
			Sei $E$ eine Einschränkung\\
			$X$ ist \textbf{Möglichkeitsraum} unter $E$ $\Leftrightarrow \forall x\in X: E(x)$
			\subsubsection{Unmöglichkeitsraum}
				$X^C = \{x \in X: E(x)\neq 0\}$, das Komplement von $X$, ist der
				\textbf{Unmöglichkeitsraum}
			\subsubsection{Spezialfall}
				Ist der Entscheidungsraum $X = M$, so kann man
				Möglichkeitsraum und Entscheidungsraum als Begriffe miteinander
				vertauschen.
		\subsection{Definition: möglich}
			Sei $X$ eine Entscheidungsraum\\
			Sei $E$ eine Einschränkung\\
			Sei $M$ der Möglichkeitsraum von $E$\\
			$x$ ist \textbf{möglich} unter $E$ $\Leftrightarrow$ $x \in M$
		\subsection{Einschränkung lässt Freiheit zu}
			Sei $E$ eine Einschränkung und $M$ sein Möglichkeitsraum\\
			$E$ lässt Freiheit zu $\Leftrightarrow$ $|M|>1$
		\subsection{Rückblick}
			Nun wurde einiges über Möglichkeiten definiert. Dies wird
			benötigt, damit man Einschränkungen auf Freiheit setzen kann.
			So kann ein Ding frei sein, jedoch eingeschränkt in
			dem Rahmen seiner Möglichkeiten.
		\subsection{Beispiel}
			Sei $X=\mathbb{R}$\\
			Sei $E:X\Rightarrow \{0,\pm 1\},\> x\mapsto |sin(x)|$ eine Einschränkung\\
			Der Möglichkeitsraum von $E$ ist $\{x \in X | |sin(x)|=0\} = \{n\pi | n\in \mathbb{Z}\}$
			Somit kann, wenn $x \in X$ frei unter der Einschränkung $E$
			gewählt werden soll, nur $x\in \{n\pi | n\in \mathbb{Z}\}$
		\subsection{Erkenntnisse}
			Bisher kann man einige der Fragen alleine mit diesen
			Definitionen beantworten.
			\subsubsection{Was ist Freiheit}
				Freiheit ist die Fähigkeit, ein beliebiges Element aus
				einem Möglichkeitsraum zu wählen.\\
				(jedoch stellt sich die Frage, was ``Wählen'' bedeutet)
			\subsubsection{Woraus entsteht Freiheit}
				Freiheit entsteht, wenn es einen Entscheidungsraum
				oder Möglichkeitsraum gibt, welcher eine Kardinalität
				größer als 1 hat.\\
				Also, es gibt mehr Entscheigungen/Möglichkeiten als eine.
			\subsubsection{Was macht Freiheit}
				Freiheit erlaubt es, Wahlen in einem Möglichkeitsraum
				zu treffen.
			Für die restlichen Fragen müssen zuerst weitere Fragen
			beantwortet werden. Die Frage, die dabei wohl am meisten
			heraussticht ist, was ``Wählen'' denn genau bedeutet.
	\section{Freiheit}
		\subsection{Definition: Freiheit}
			Sei $M$ ein Möglichkeitsraum.\\
			$f: \mathcal{P}(M) \rightarrow M$ ist eine Freiheit, wenn
\end{document}
